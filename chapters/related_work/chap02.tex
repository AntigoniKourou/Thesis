%
% File: chap02.tex
% Author: Antigoni Kourou
% Description: Literature review.
%
\let\textcircled=\pgftextcircled
\chapter{Current state of knowledge}
\label{chap:02}

\initial{T}he task of extracting opinions from customer reviews is an issue, which has the attention of research communities for more than one and a half decade. With the continuous growth of Internet use, numerous social networks, blogs, news reports, forums, e-commercial websites and other platforms serve as an open-space for expressing opinions. The domain of interest varies from general public events to political campaigns as from product choices to marketing strategies. Sentiment analysis (SA) is the study of opinions, emotions and attitude towards an entity expressed in free text \cite{ravi2015survey}. Sentiment analysis itself is a problem that includes several inter-related aspects, such as subjectivity classification, polarity determination, review usefulness measurement, opinion spam detection and so further. Due to the lack of standards in the methods used and the complexity of the problem, it is difficult to portray and categorize all the views presented in the literature. From 162 research articles covered in \cite{ravi2015survey} during the time frame 2002-2015, my main focus are the feature-based opinion mining systems. 

The reviewed systems base their extraction on statistical methods, unsupervised learning methods or ontologies. Statistical methods typically apply some rules on the frequency of nouns to find explicit features,  unsupervised learning method group the explicit features and the similarities by measuring distributional properties of words and ontologies provide a pre-set list of features for the systems to be based.
On the other side, there are mainly two approaches to polarity determination: machine learning and lexicon based methods. The level of sentiment analysis here differs from document level, sentence level or word level. The examined systems vary from each other on the methods used for feature identification, the granularity of sentiment analysis, the methods used for polarity detection or the way the output is presented. Due to the fact that this paper makes use of an ontology based approach, the examined systems are grouped into non-ontology based and ontology based, although it is hard to compare the performance of these systems due to the lack of standard test data and methods.
%-------------------------------------------

Starting from non-ontology based approaches,  Liu and Hu introduced in 2005 Opinion Observer, a lexicon-based system for sentiment analysis of opinions \cite{liu2005opinion,hu2004mining}. Their system extracts product features by using association rule mining, explicitly the Classification Based on Association (CBA) algorithm. The system only uses adjectives as opinion words and assigns their polarity based on WordNet dictionary. The polarity of an opinion expression is determined in sentence level based on the number of positive/negative words. The final output are stored in a database in the form of \textit{(feature, number of positive expressions, number of negative expressions)}. 

In 2007 two other systems are proposed, namely Red Opal \cite{scaffidi2007red} and the famous OPINE \cite{popescu2007extracting}. Red Opal aims to score product features based on customer reviews. It uses probability-based heuristics for identifying frequent nouns and noun phrases for feature extraction and it is the only systems which makes use of review stars for defining the sentiment of features. In other words, the rating a reviewer would give are not calculated as an overall score but are assigned to specific features mentioned in their own reviews. A user interface is used for offering to the customers a ranked list of products based on the selected features. 

The second approach, OPINE \cite{popescu2007extracting} is based on KnowItAll information extraction system which is domain-independent. This approach  extracts explicit product features using the Pointwise Mutual Information (PMI) between phrases. It uses explicit features to identify potential opinion phrases based on assumption that "an opinion phrase associated with a product feature will occur in its vicinity on syntactic parse tree" \cite{popescu2007extracting}. After the extraction of the opinion expression, the semantic orientation of opinion words is determined by using relaxation labeling, an unsupervised classification technique.  

Weakness Finder \cite{zhang2012weakness} is one of the well-known proposal of 2012. It aims to help the identification of features, and group these features into different aspects by using explicit and implicit features grouping methods, then judge the polarity of each sentence by using sentence-level sentiment analysis. The methods used for feature extraction are both collocation statistic based methods and lexicon-based on HowNet dictionary, thus a hybrid approach. The sentiment of features is detected in sentence level as positive, negative or neutral extending Liu's work, Opinion Observer system \cite{liu2005opinion}.

Bagheri et al. proposed in 2013  a novel unsupervised and domain-independent model for detecting explicit
and implicit aspects in reviews for sentiment analysis \cite{bagheri2013care}. In the model, first a generalized method is proposed to learn multi-word aspects and then a set of heuristic rules is employed to take into account the influence of an opinion word on detecting the aspect. Second a new metric based on PMI and aspect frequency is proposed to score aspects with a new bootstrapping iterative algorithm, works with an unsupervised seed set. Utilizing extracted polarity lexicon, the approach maps each opinion word in the lexicon to the set of pre-extracted explicit aspects with a co-occurrence metric. The output would look like in the example \textit{(great - phone, price); (good - battery life, sound quality)}.

The first proper system using an ontology-based approach is proposed in 2012. Fuzzy domain ontology sentiment tree (FDOST) \cite{liu2012toward} aims to discover sentiment polarities over product features based on ontologies, which unites traditional natural language processing (NLP) techniques with sentiment analysis processes and Semantic Web technologies. The authors first construct the hierarchical relationships among product's attributes, product's attributes and corresponding sentiments through fuzzy sets. Then they extend the hierarchy of FDSOT, including the extraction of sentiment words, product features and the relations among features. Finally, the polarity weights of features are assigned making use of Double Propagation. The output of the FDOST are the features and the polarity weights, for example \textit{\{cheap, +1626.729\}} reveals that there are 1626 consumers expressing positive opinion about the price.

In 2014 Penalver et al. proposed an another "novel" ontology based methodology for feature extraction \cite{penalver2014feature}. Their system has four stages: a) feature identification based on ontology mechanism; b)polarity assignment to each feature based on SentiWordNet and the relative position in each user's opinion; and c) a new approach for opinion mining based on vector analysis. The resulting polarity value is an Euclidean vector with three coordinates (x, y, z) for each feature counting the number of positive, negative and neutral opinion respectively. The sum of these vectors will be the global polarity expressed by a user or for a certain feature. However the sentiment classification of all the opinions is based on document level, meaning that for each feature at least 100 sentences are needed to extract its sentiment.

Lastly, in 2015 is introduced Type-2 fuzzy ontology called T2FOBOMIE, a novel extraction and opinion mining system based \cite{ali2015type}. The system reads the  customers' full-text query for hotel search and reformulates it for extracting the user requirement into the format of a proper classical full-text search engine query. The proposed system retrieves targeted hotel reviews and extracts feature opinions from reviews using a fuzzy domain ontology. T2FOBOMIE uses a classical extensible markup language (XML)-based ontology, Protege OWL. The sentiment of features is identified in lexicon-based method, making use of SentiWordNet for detecting sentiment score and then grouping it into \textit{positive, positive, neutral, negative, negative}. The results are presented with the help of an user interface which requires the customer to write his full-text query and the output of the system would be a list of hotels, the corresponding sentiment polarity and a link to the website. 

\begin{table}[h!]
\footnotesize 
\centering
\begin{tabular}{|m{1.8cm}||m{2.5cm}|m{2.7cm}|m{1.2cm}|m{3cm}|m{1.15cm}|}

\hline
\centering {\textbf{SYSTEM}}  & \centering {\textbf{FEATURE EXTRACTION}} & \centering {\textbf{SENTIMENT DETECTION}} & {\textbf{DOMAIN}}  & \centering{\textbf{OUTPUT}} & {\centering\textbf{LANG}} \\[0.7cm]

\hline
\centering {Opinion Observer (2005)}  & \centering {Association miner, CBA} & \centering {Lexicon-based (WordNet)} &  {Product}  & \centering{Feature, \# of positive expression, \# of negative expression} & {English}\\ \hline

 \centering {Red Opal (2007)} & \centering {Probability-based heuristics} & \centering  {Assign star rating} & {Product}  &\centering{User interface of ranked results}& {English} \\ \hline
 
 \centering {OPINE (2007)} & \centering {Unsupervised, Web PMI} & \centering {Relaxation labeling} & \centering { Ind. }  & \centering {List of sentiment sentences} & {English}\\ \hline
 
\centering  {Bagheri et al. (2013)} & \centering {PMI, DBA} & \centering {Bootstrapping} & {Product} & \centering{A co-occurrence matrix: (feature; opinion words)}  & {English} \\ \hline

\centering {Weakness Finder (2012)} & \centering {Collocation statistics} & \centering {Lexicon-based (Hownet)} & {Product}  & \centering{List of features with negative sentiment} & {Chinese} \\ \hline
\centering {FDSOT (2012)} & \centering {Fuzzy set, Ontology-based} & \centering {Double Propagation} & {Product}  &\centering {Set feature;polarity} & {Chinese}  \\ \hline

\centering  {Penalver et al. (2014)} & {Ontology-based} & \centering {Dictionary-based (SentiWordNet)} & {Movie} &  Euclidean vectors of polarity  & {English, Spanish}\\ \hline

\centering  {T2FOBOMIE (2015)} & \centering {Ontology-based} & \centering {Lexicon-based (SentiWordNet} & {Hotel}  & \centering{List of hotels with positive/negative polarity on feature} & {English}\\ \hline

\centering  {The proposed approach} & \centering {Ontology-based \& lexicon-based (WordNet)} & \centering  {Rule-based and lexicon based (VADER)} & {Hotel}  & \centering {Matrix of discrete scores per each sentence and features} & {English} \\ \hline
\end{tabular}
\caption{Comparison of examined systems on feature-based opinion mining}
\label{comparison}
\end{table}

Table \ref{comparison} gives an overview of the comparison within these systems based on methods they use for feature extraction, sentiment detection, domain of usage, language and output format.

\section{Implications of literature review}
The examined approaches are mostly found for English or Chinese reviews. A big limitation of the systems based on statistical and unsupervised learning methods is that they ignore features with low frequency of occurrence. This fact is not anymore an issue for the ontology-based systems, however they vary on the methods used for building the ontology as there lacks any unified, standard, multi-domain ontology. The sources used for building the ontology of my proposed approach will be described in the Methodology chapter.
Secondly, the examined systems detect the sentiment of features based on positive, negative and neutral polarity and not in discrete scores. Grouping the polarity in this form guarantees better scores in the accuracy of algorithms, however it does not leave room for further analysis with the data.
In addition to this, the third limitation is the way of assigning sentiment to features. Depending on sentence or document level, an overview of the number of positive/negative opinions, the sentiment sentences itself or either their summaries, do not contribute in getting deeper insights from text, neither reduce the workload of the customers and service providers for analyzing the data.
Thus, this paper proposes the use of estimated discrete scores of sentiment per feature in order to get deeper insights in customers opinion from free text and align this with the quantitative data retrieved from the feedback system. This proposal will be explained in the following chapter.
