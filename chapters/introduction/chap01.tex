%
% File: chap01.tex
% Author: Victor F. Brena-Medina
% Description: Introduction chapter where the biology goes.
%
\let\textcircled=\pgftextcircled
\chapter{Introduction}
\label{chap:intro}
%
% General description of ICTs growth, etourism term and feedback systems
%
\initial{T}he exponential growth of Information and Communication Technologies (ICTs) has played a great role in the development of tourism industry. Electronic tourism (e-tourism) is the application of ICTs for tourism purposes, including the digitalization of all its processes and value chains \cite{buhalis2003etourism}. Increasingly, ICTs provide users with access to many sources of information and has eventually affected the consumer behavior in tourism industry \cite{mills2004handbook}. Nowadays there is a huge variety of web services which provide a great complexity and diversity of recommended offers for meeting the demands of travelers. However, the personalized consumption patterns and individualistic lifestyles make it incredibly difficult for the service providers to anticipate tourists' behavior \cite{niemann2008enhancing}. For decision makers to successfully understand the requirements, needs, desires and preferences of the customer, detailed information has to be obtained. Travelers, making use of the ICT tools that facilitate the information retrieval and decision making processes, have now direct access to all types of information provided by tourism agencies, companies, marketers, enterprises or other users. Furthermore, ICTs and Internet have transformed e-tourism markets from customer-centric to customer-driven \cite{buhalis2011tourism}, meaning that users play a major role in creating and sharing traveling information through blogs and review websites. Online feedback mechanisms, also known as reputation systems, \textit{have emerged as a viable mechanism for fostering cooperation among strangers in such settings} \cite{dellarocas2003digitization}. Examples of these systems, for instance TripAdvisor, Booking.com or AirBnb, after each trade encourage both parties to give feedback about their trading partner based on their own experience. 
\section{Problem statement}
%
% How feedback is collected and analyzed 
%
The most common types of consumer generated feedback are ratings from 1-5 stars and general text comments. An important separation exists between the distinct role of text comments as tacit knowledge and ratings as explicit knowledge and the ways they are analyzed. For online marketplaces to succeed, their feedback technologies must be able to not only collect users feedback, but to properly analyze it and utilize for decision making purposes \cite{pavlou2006nature}. However, current online travel systems, aiming to assist the consumer in finding suitable offers, filter the information based on location-price factor and on the overall ratings accumulated from the feedback system, meaning that text reviews are revealed for the public to read but they do not directly affect the overall analysis. Focusing on solely numerical ratings and ignoring the importance of text feedback leads to two major issues for feedback systems. 
%
% The issues with current analysis
% ---- BIAS -----
%

First, many academic papers on online reputation systems and building trust in the online marketplace report the existence of bias on online reviews \cite{bolton2013engineering,dellarocas2008sound,dini2009buying,fradkin2016bias,ghose2011estimating,resnick2006value}, thus reducing the bias of these systems is an important issue towards a more efficient online feedback system. Utilizing only biased ratings does not necessarily mean that the top result is the most suitable option for a certain user considering the personal requirements. On the other hand, the analysis of ratings does not indicate much information for the service providers, who aim to acquire knowledge on how to improve their services. In order to gain detailed insights from the feedback system, some service providers including Airbnb, ask its users to rate not only the overall quality of the listing, but also six accommodation features. However, since the overall ratings are biased \cite{fradkin2016bias}, how do we make sure that the ratings for specific features are objective? From this point of view, the bias on quantitative data of the system raises the issue of reliability on the system itself.
%
% ---- HIDDEN FEATURES -----
%

Second, the incompleteness of the analysis based solely in quantitative data leads to the need for complementary analysis. Customers' needs are considered multi-dimensional and difficult to measure on discrete scales such as ratings \cite{luo2005information}, therefore a customer has to extract the needed information from different sources and types of information provided by agencies, companies or other users. According to \cite{pavlou2006nature}, text comments are particularly interesting for the audience as a new trust-building means in online marketplaces by revealing hidden knowledge, which is often underestimated from their owners and cannot be described by negative/positive ratings. Furthermore, \cite{fradkin2016bias} suggests that text opinions influence the decision making process even when the ratings are high. In the Airbnb feedback system a negative rating is followed by a text in 45\% of the cases, which implies the great power of text analysis for discovering deeper insights for the listing \cite{fradkin2016bias}. Acknowledging the importance of text comments, some feedback systems often offer summaries to all text comments, which mostly consist on a bunch of most used words. However, this bag of words does not necessarily cover the features that a certain user is interested in, neither the features that need to be improved. The users or service providers still have to read all the text comments related to the feature, meaning that it still does not reduce much of the  work. A survey by \cite{pavlou2006institutional} asked the respondents to indicate how many feedback comments they examined before each online transaction. The result showed that 81\% reported examining 25 comments (one webpage), 5\% viewed 50 comments, 11\% more than 50 ones, and only 3\% did not examine any text comments. These findings reveal that despite the importance of text feedback to the users, it is difficult for them to access the meaning of numerous text comments \cite{pavlou2006institutional}. Given this situation, the average human reader will have difficulties on identifying and extracting the relevant information from the opinions in them. Automated analysis systems are thus needed \cite{liu2012sentiment}.

\section{Research question}
Natural language processing (NLP) enable computers to derive meaning from any human written input, including their opinions. The NLP methods for doing so fall into the category of sentiment mining methods, known also as \textit{opinion mining}. Examples of their application include mainly the movie rating systems (Netflix, IMDb) and the product rating systems (eBay). However, the importance of extracting sentiment of features from comments, besides their overall positive/negative sentiment is often ignored in the literature. This research proposes the implementation of feature-based opinion mining methods from complementing the analysis of customer feedback in e-tourism and accommodation market. The proposed approach uses an ontology based approach combined with sentiment mining techniques for generating opinion scores for each accommodation feature mentioned in the reviews of a feedback system. This solution deals with the two issues mentioned above, bias of ratings and the need for complementary text analysis for feature extraction. From its point of view, both issues can be brought together as one, since text analysis is believed to contribute to a better rating systems by reducing its bias \cite{fradkin2016bias}.

This paper answers firstly the question of \textit{how can opinion mining methods be aligned with the quantitative data analysis in order to enhance focus on customer feedback and produce detailed analysis results.} By estimating sentiment scores for the text reviews, the pipeline transforms the text data into discrete quantitative form, which can easily be analyzed for different purposes. Some of the implications of the analysis are treated in the next sections. In addition to this analysis, it is important to find out \textit{how good can feature-based opinion mining estimate the quality of accommodation features in e-tourism feedback systems,} which measures the reliability of the analysis. By answering these two questions, the purpose of the paper is to offer to service providers a new reliable approach on enhancing focus on customers, based on users' generated content.

\section{Research rationale and structure}
%
% The rationale is missing
%
To provide an answer to the research questions, this paper is organized in four(?) chapters. The first part of the paper introduces the importance of gathering and analyzing customer feedback in e-tourism. The current state-of-the-art of feature-based opinion mining used for analyzing customer feedback is covered in the second part. The literature review includes articles published in a time frame of the last ten years and it leads the reader to the approach proposed for filling the literature gaps. The proposed approach is discussed by explaining each step of the pipeline developed for this research. The fourth chapter covers the methodology used in the research, from collecting the data to the analysis of the output. In chapter five, the whole proposed approach is evaluated based on human logic. Afterwards the results are presented and discussed. Finally, the paper concludes with implications of the proposed approach, limitations and further work to be done. 
