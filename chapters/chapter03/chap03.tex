%
% File: chap03.tex
% Author: Antigoni Kourou
% Description: Methodology used for the research.
%
\let\textcircled=\pgftextcircled
\chapter{Methodology}
\label{chap:methods}

\initial{H}ere will go the methodology used for the research. First, I will explain the Airbnb case, including: why there is a bias in this system, how does the rating works and how will I do the research using their data. 
\section{The Airbnb case}
\textit{Preliminary only:} Airbnb is a large online marketplace for accommodations, therefore reputation and trust is very important for the transactions on Airbnb. Before discussing the issues of the system, let’s take a look on how it works. The Airbnb feedback system is two sided, meaning that both guests and hosts have 14-30 days to review each other. The users are encouraged by email, login notifications or other forms of reminders to rate their experience and the feedbacks are revealed simultaneously when both the guest and the host have rated each other. The structure of feedback consist of three parts: firstly, the users is asked to give a general rating from 1-5 stars and a general comment, secondly the users rate six defined categories (Accuracy, Communication, Cleanliness, Location, Check in and Value) and the average ratings are published only after three reviews; and lastly the users are asked if they would recommend the listing or not. A recent research on this system implies that the system suffers from the bias on reviews \cite{fradkin2016bias}. In addition the research suggest that the bias on the system is caused mainly because of three reasons. First, ratings are the only data on feedback analysis of the system, meaning that text reviews are revealed for the public to read but they do not affect the overall rating of the listing. Second, based on the surveys of Airbnb itself, the non-reviewers tend to have worse experience than the reviewers and the third reason is the fact that the system does not offer an option for the feedback to remain anonymous as most of the users prefer.

\section{Data mining}
\subsection{Data collection}
This subsection will describe the characteristics of data that will be used for research. In details how each property of the data scraped will be used, name of databases and so on.
\subsection{Data pre-processing}
This subsection will explain how the data will be cleaned up. Here important to mention is: The use of only english text (how will be filtered), how will deal with negation, how will deal with bad english etc. IMPORTANT PART (the unsolved problems need to be mentioned in the limitations.)
\subsection{Data analysis}
Here the chosen algorithm will be described step by step. 

\section{Evaluation of the chosen algorithm}
This section will explain that in order to check the efficiency of the chosen algorithm, it is tested by humans. A sample of text reviews which are mined using the algorithm are then given to humans to find out the percentage of the precision of the algorithm.The setting will consist of an online survey, in which for a given text each person will have to identify the sentiment for a list of aspects. If the aspect is not mentioned in the text, then it will have no rating. The rating willl be from 1-5. The text of these reviews will be in Appendix as well as the answers of each person, who filled the survey.
\subsection{Participants}
\subsection{Survery questions}
\subsection{Analysis}
\section{Calculating the bias of the system}
IMPORTANT - Here will be the formulas how we can calculate the bias on the online reviews. This section can be totally based on the Airbnb research, which indicates the bias in the system.