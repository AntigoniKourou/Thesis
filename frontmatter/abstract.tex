%
% File: abstract.tex
% Author: V?ctor Bre?a-Medina
% Description: Contains the text for thesis abstract
%
% UoB guidelines:
%
% Each copy must include an abstract or summary of the dissertation in not
% more than 300 words, on one side of A4, which should be single-spaced in a
% font size in the range 10 to 12. If the dissertation is in a language other
% than English, an abstract in that language and an abstract in English must
% be included.
\let\cleardoublepage\clearpage
\chapter{Abstract}
\begin{SingleSpace}

\initial{O}nline {\color{red}{reviews are an important source for both customers, to obtaining information before making purchase decisions, and businesses to improve the quality of their services. Opinion mining is an active field of research that makes use of Natural Language Processing (NLP) techniques for mining the huge amount of online text reviews. This paper aims to enhance the focus on customers by analyzing their opinions and aligning the results with the quantitative feedback received. The setting of this research is the dataset of Airbnb reviews, a giant online marketplace for vacation rentals. }} Explicitly, 
\todo[color=green!40]{I suggest to skip the first paragraph, and just add a sentence that you do research on AirBnb, and in particluar compare ratings based on stars and the sentiment expressed in the reviews}
this paper proposes the use of feature-based sentiment mining for estimating discrete score for each accommodation feature, based on online reviews. The proposed approach is composed by three stages: (i) feature extraction from full text reviews based on accommodation ontology; (ii) sentiment detection for each feature on sentence-level and hybrid-based method, including both rule-based and lexicon approaches; and (iii) data analysis of discrete sentiment scores per feature and overall. The aim of this proposal is to show how insights from text content can be aligned with the quantitative data of feedback systems for different analysis purposes. The two main analysis directions include firstly the comparison of ratings generated by text with the ratings of Airbnb system and its bias issue, and secondly the extraction of features and their corresponding sentiment based on text. The proposed approach is evaluated towards human's logic and it reaches the accuracy 77.8\% in feature identification and 70.1\% in sentiment score detection. These results are quite comparable with the state of knowledge on feature-based opinion mining.
\end{SingleSpace}
\clearpage