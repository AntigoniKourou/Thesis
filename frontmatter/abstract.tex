%
% File: abstract.tex
% Author: V?ctor Bre?a-Medina
% Description: Contains the text for thesis abstract
%
% UoB guidelines:
%
% Each copy must include an abstract or summary of the dissertation in not
% more than 300 words, on one side of A4, which should be single-spaced in a
% font size in the range 10 to 12. If the dissertation is in a language other
% than English, an abstract in that language and an abstract in English must
% be included.
\let\cleardoublepage\clearpage
\chapter{Abstract}
\begin{SingleSpace}
\initial{T}he increased use of Internet and Information and Communication Technologies (ICTs) have affected radically the tourism industry. Current online travel systems, aiming to assist the consumer in finding the suitable offers, provide a great complexity and diversity in offers to fulfill their customers' needs. However, its is incredibly difficult for service providers to understand their customers needs and preferences. ICTs and Internet have transformed the markets to cusotomer-driven \cite{buhalis2011tourism}, meaning that the users are co-creators of traveling information through blogs, forums, social media and feedback systems.  Customer-created content is found mainly in the form of free-text opinions, which require a tremendous amount of work to be manually analyzed and get the proper insights. The tools for automating this analysis fall into the category of Natural Language Processing (NLP), explicitly sentiment analysis. 

This paper proposes the use of feature-based sentiment mining for estimating discrete score for each accommodation feature, based on reviews of feedback system. The proposed approach comes in three stages: (i) feature extraction from full text reviews based on accommodation ontology; (ii) sentiment detection for each feature on sentence-level and hybrid-based method, including both rule-based and lexicon approaches; and (iii) data analysis of discrete sentiment scores per feature and overall. The aim of this proposal is to show how insights from text content can be aligned with the quantitative data of feedback systems for different analysis purposes. These insights include: comparison of ratings generated by text with the ratings of the system and the bias issue, extracting important features and their sentiment, querying results based on individual preferences, finding the weakness of the service and so further. The proposed approach is evaluated towards human's logic and it reaches the accuracy 77.8\% in feature identification and 70.1\% in sentiment score detection. These results are quite comparable with the state of knowledge on feature-based opinion mining.
\end{SingleSpace}
\clearpage