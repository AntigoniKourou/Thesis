%
% File: abstract.tex
% Author: Antigoni Kourou

% be included.
\let\cleardoublepage\clearpage
\chapter{Abstract}
\begin{SingleSpace}

\initial{O}nline reviews are an important source of obtaining information for both businesses and their customers. In the last two decades the amount of text data on the Web has increased significantly, therefore automatic tools for analyzing these data exist. Most of the tools fall into the category of sentiment mining, an active field of research that makes use of Natural Language Processing (NLP) techniques for detecting reviewers' sentiment and opinions from free text. Some of the existing tools are gaining popularity also in the business circles, however it is still unclear how their use can be unified with the analysis of feedback received in quantitative form. Thus, this paper proposes the use of feature-based sentiment mining for estimating discrete opinion scores of features, based on online reviews. The setting of this research is the dataset of Airbnb reviews, a giant online marketplace for accommodation rentals. The proposed approach of this paper is composed of three stages: (i) feature extraction from full text reviews based on accommodation ontology; (ii) sentiment detection for each feature on sentence-level and hybrid-based method, including both rule-based and lexicon approaches; and (iii) data analysis of discrete sentiment scores per feature and listings. The results of the analysis show that feature-based sentiment mining is a reliable method for discretely estimating the ratings of listings and features from text reviews, in comparison with the Airbnb values. Furthermore, this research shows a tendency of Airbnb system to systematically highly rate the accommodation rentals, which can be decreased by using text reviews for complementary analysis of the quantitative feedback. The evaluation of the pipeline demonstrates high accuracy on detecting average sentiment scores per features or listings. For the task of feature extraction, the pipeline achieves on average 77.7\% precision and 79.2\% recall, meaning that it is comparable with the current state of art in the field. 
\end{SingleSpace}
\clearpage