%
% File: abstract.tex
% Author: V?ctor Bre?a-Medina
% Description: Contains the text for thesis abstract
%
% UoB guidelines:
%
% Each copy must include an abstract or summary of the dissertation in not
% more than 300 words, on one side of A4, which should be single-spaced in a
% font size in the range 10 to 12. If the dissertation is in a language other
% than English, an abstract in that language and an abstract in English must
% be included.
\chapter{Abstract}
\begin{SingleSpace}
\initial{T}he rapid development of Information and Communication Technologies (ICTs) have affected radically the tourism industry. The term e-tourism is adopted to mirror all the applications of ICT in the digitalization process of all tourism services and its value chains \cite{buhalis2003etourism}. Nowadays there is a huge variety of web services that provide a great complexity and diversity of offers in order to meet the demands of travelers. However, the individualistic consumption and lifestyles make it incredibly difficult for the service providers to anticipate users' behavior \cite{niemann2008enhancing}. Consumers needs are considered multi-dimensional and difficult to measure on discrete scales \cite{luo2005information}, therefore a traveler has to consider several sources and types of information provided by companies, agencies or other travelers before making the booking decisions. This means a big part of the work on finding the relevant information is totally up to the consumer. 
Current online travel systems, aiming to assist the consumer in finding the suitable offers, base their ranking and recommendations on the location-price factor and on the overall ratings accumulated from the feedback systems. To fulfill their requirement, the travelers tend to check themselves the comments of each item listed by the search engine. This task is not only time consuming but also inefficient when the user only for finding opinions about a certain feature has to read hundreds of comments. The service providers often offer summaries to all feedback comments, which mostly consist on a bunch of most used words. However, except of the fact that this bag of words may not contain the features that the user is looking for, it also does not compare the items of the search result with each other, meaning it still does not reduce much of the travelers' work.  This paper proposes the use of feature-based sentiment mining for measuring and ranking the search results of e-tourism platforms based on accommodation features which reflect the specific consumer requirement. Even though these requirement are considered difficult to measure, a discrete score can be generated for their quality based on the opinion of other users.
\end{SingleSpace}
\clearpage