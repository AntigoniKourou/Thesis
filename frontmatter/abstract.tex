%
% File: abstract.tex
% Author: V?ctor Bre?a-Medina
% Description: Contains the text for thesis abstract
%
% UoB guidelines:
%
% Each copy must include an abstract or summary of the dissertation in not
% more than 300 words, on one side of A4, which should be single-spaced in a
% font size in the range 10 to 12. If the dissertation is in a language other
% than English, an abstract in that language and an abstract in English must
% be included.
\let\cleardoublepage\clearpage
\chapter{Abstract}
\begin{SingleSpace}
\initial{T}he rapid development of Information and Communication Technologies (ICTs) have affected radically the tourism industry. E-tourism refers to the use of ICT applications in the digitalization process of all tourism services and its value chains \cite{buhalis2003etourism}. Nowadays there is a huge variety of web services that provide a great complexity and diversity in offers, therefore a traveler has to consider several sources and types of information provided by companies, agencies or other travelers before making the booking decisions. Current online travel systems, aiming to assist the consumer in finding the suitable offers, base their ranking and recommendations on the location-price factor and on the overall ratings accumulated from the feedback systems. Other customer needs are considered multi-dimensional and difficult to measure on discrete scales \cite{luo2005information}.This paper proposes the use of feature-based sentiment mining for estimating a discrete score for each accommodation feature, based on reviews of feedback system. The main idea of the proposal is to offer to the user the possibility of ranking their search results based on personal needs, represented by accommodation features. The proposal uses an ontology based approach for extracting the relevant features and afterwards detects the sentiment expressed for the chosen feature in the text reviews of an item. **Evaluation missing**
\end{SingleSpace}
\clearpage