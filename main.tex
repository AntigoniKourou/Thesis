\documentclass[a4paper, 11pt]{article}
\usepackage{comment} % enables the use of multi-line comments (\ifx \fi) 
\usepackage{lipsum} %This package just generates Lorem Ipsum filler text. 
\usepackage{fullpage} % changes the margin

\begin{document}

\section*{Problem Statement}

\section*{Investigation/Research}


\section*{Alternative Solutions}


\section*{Optimum Solution}
% to comment sections out, use the command \ifx and \fi. Use this technique when writing your pre lab. For example, to comment something out I would do:
%  \ifx
%	\begin{itemize}
%		\item item1
%		\item item2
%	\end{itemize}	
%  \fi

Here will go the introduction. \textit{Preliminary example}: Sharing economy is an important phenomenon grown exponentially over the last decade. By utilizing information technology, individuals can reuse, distribute and share their excess capacity of goods and services. However, building trust in these online marketplaces is the key to the adoption of the sharing economy \cite{owen2014trust}. All the markets require some minimum amount of trust, but for Internet markets this is a particular challenge considering the fact that trades are typically anonymous, geographically dispread and executed sequentially. To incentivize trustworthiness, online markets employ reputation based "feedback systems" allowing traders to post information about past experiences. Examples of these systems include the traditional giants eBay and Amazon, to continue with the new big players such as Airbnb and Uber.

(missing part)
Sentiment mining is the computational study of opinions, sentiments, evaluations, attitudes, appraisal, affects, views, emotions, subjectivity, etc., expressed in text. This text can be found in reviews, blogs, discussions, news, comments, feedback and so on. This analysis can also be mentioned as opinion mining and it refers to opinions gathered first from organizations internal data (customer feedback from emails, call centers), secondly from news and reports and thirdly based on the word-of-mouth on the Web (personal experiences, issues, reviews, social networking, blogs, forums etc). 

----> Sentiment analysis application? Business \& orgs, individuals ...

Since early 2000, sentiment analysis has grown to be one of the most
active research areas in natural language processing. It is also widely studied
in data mining, Web mining, and text mining. In fact, it has spread from
computer science to management sciences and social sciences due to its
importance to business and society as a whole. In recent years, industrial
activities surrounding sentiment analysis have also thrived. Numerous
startups have emerged. Many large corporations have built their own inhouse
capabilities. Sentiment analysis systems have found their applications
in almost every business and social domain.

With the explosive growth of social media (e.g., reviews, forum discussions,
blogs, micro-blogs, Twitter, comments, and postings in social network sites)
on the Web, individuals and organizations are increasingly using the content
in these media for decision making. Nowadays, if one wants to buy a
consumer product, one is no longer limited to asking one’s friends and
family for opinions because there are many user reviews and discussions in
public forums on the Web about the product. 
Each site
typically contains a huge volume of opinion text that is not always easily
deciphered in long blogs and forum postings. The average human reader will
have difficulty identifying relevant sites and extracting and summarizing the
opinions in them. Automated sentiment analysis systems are thus needed.\cite{liu2012sentiment}

information obtained from IDS
could be treated as a proxy measure of sociological or economical phenomena (e.g. Google
Trends). Moreover, there is a lack of statistical literature directly connected with estimation
problems related to new data sources. The results of the study suggest that the existing denitions and methodology, which are
valid for existing statistical data sources, should be adopted or revised to deal with new data
sources. \cite{berkesewicz2015representativeness}

While ‘manufactured’ trust-building arguments by Internet stores may be biased, text comments written by buyers in online auction marketplaces are likely to be objective, impartial, and unbiased. The tacit nature of feedback comments can convey the notion that a seller has previously acted in an outstanding fashion to pursue a buyer’s best interests, or acted in an abysmal manner to exploit a buyer’s vulnerabilities.
Most important, a distinct survey item asked the respondents to indicate how many feedback comments they examined for the seller they purchased from. 81\% reported examining 25 comments (one webpage), 5\% viewed 50 comments, 11\% more than 50 ones, and only 3\% did not examine any text comments. This suggests that the evaluation of the first 25 comments in a seller’s feedback profile is likely to provide representative information about each seller that is typically examined by buyers. \cite{pavlou2006institutional,pavlou2006nature} 
%=======
\section{Customer Focus Theory}
\label{sec:CFTH}

When offering a service or product, it is essential to understand what consumers' preferences are and how can we shape the offerings accordingly to create value. Customer Focus Theory is one of Information Studies theories, which specifically offers a guide on how to put the focus on the customer. This theory should be explained according to the context of my research (where it is placed and adds value on the framework) as shown in Figure. 
--- Narrow down to "Receiving and utilization of customer feedback" and the accordance with the research topic.

% This is a requirement of Business Information Studies track: Base and support the thesis research on one (or more) of Information Studies theories. Link: http://is.theorizeit.org/wiki/Main_Page 

\section{Online Feedback systems}


\subsection{Principles of OFS and game theory}

Various mechanisms have been designed and utilized
in C2C auction markets to promote trust and reduce risk.
For instance,
eBay's “Feedback Forum” is a form of community enforcement.
In eBay, after each trade, both buyers and
sellers are encouraged to leave comments about their
trading partners based on their experience. Comments
about traders are kept under each trader's profile, and can
be accessed by everyone who visits eBay. This way, the
system tries to deter dishonest behavior by conveying
facts and opinions about past trades. Kollock [15]
conceptually summarizes online reputation systems and
concludes that their effectiveness to manage the risks of
unsecured trades seems to be impressive. Resnick et al.
[20] review the online reputation systems and argue that
the reputation systems appear to perform reasonably well
despite their theoretical and practical difficulties. \cite{yang2007effects}


\subsection{Characteristics of most popular OFSs}
\label{subsec:popularOFS}

This subsection will explain how online feedback systems work. In addition, it will describe, categorize and point out the differences between the typical types of OFS used by the most successful online marketers (Ebay, Amazon, Yelp, TripAdvisor and Airbnb). 

\subsection{Methods for analyzing customer feedback}
\label{subsec:feedbackmethods}
This subsection will mention the methods how the customer feeback can be analyzed in order to get the most out of it. The ratings and the text reviews will be explicitely mentioned in order to create the right environment for jumping to the problem statement.

\subsection{Bias in the OFSs}
\label{subsec:bias}
Explains the phenomenon of bias in the online feedback systems and also the different types of bias noticed in them. The subsection will end up to the bias that this research aims to reduce.

\textit{Preliminary:} Many academic papers on online reputation systems and building trust in the online marketplace report the existence of bias on online reviews [1, 3, 4, 5, 6, 9], therefore reducing the bias of these systems is an important issue towards a more efficient online feedback system. 

\section{Problem statement}
\label{sec:problemstatement}
This section will clearly define the problem and the research question at the end.The first subsection focuses on the methods proposed to move toward a more efficient feedback system.

\subsection{Three ways to move toward a more efficient OFS}
\textit{Preliminary:} In 2008 the giant of online marketplaces eBay changed radically the way how their feedback system was working. Many researchers have analyzed the eBay changes \cite{fradkin2016bias,resnick2006value,bolton2013engineering,dini2009buying,dellarocas2008sound}, and findings suggest three ways to move toward a more efficient feedback system. A solution would be to mitigate to a new validated feedback system and follow it strictly. However abandoning the existing system and move to a new one requires a lot of effort and sometimes the model also does not fit considering the differences in the type of transactions. The second solution suggested is to build channels of feedback in a targeted way, for example only one side feedback or the element of anonymousness. The third solution learnt from the eBay case suggests using complementary methods for feedback analysis. This third solution offers in itself a lot of potential considering the big variety of tools for data analysis. This paper proposes text mining of reviews as a complementary method for feedback analysis in the reputation systems. 

\subsection{Feature based opinion mining}
This subsection will introduce the concept of sentiment mining and how it can be used for analysing the text of customer feedback in online feedback systems.

\subsection{Research question}
\textit{Preliminary:} Text reviews as part of the feedback in the online marketplace have a big importance for the users of
these platforms, which is often underestimated from their owners. Research in the field suggest that text
opinions influence the users’ decisions even when the rating for the listing is high \cite{fradkin2016bias}. Furthermore, the
study of the Airbnb feedback systems argues that a negative rating is followed by a text in 45\% of the cases,
which implies the great power of text mining for discovering the negative features of the listing. The methods for doing so fall into the category of sentiment based opinion mining methods. Examples of their application include mainly the movie rating systems (Netflix, IMDb) and the product rating systems (eBay).
This research proposes the implementation of sentiment based opinion mining methods as complementary for the review analysis in the online feedback systems of the accommodation market, explicitly in the Airbnb feedback system.
This research is based mostly on the question: “How can sentiment-based opinion mining methods complement the analysis of reviews in an online reputation system?” The research aims to test the methods which can effectively calculate the reputation scores of the text reviews and afterwards find the ways how these methods can be integrated with the current methods of feedback analysis. To be more precise, the Airbnb system calculates now the overall rating for a listing based on the average score of at least three reviews and this score has a proven bias (tend to be always positive). Given the fact that text reviews reveal often the negative aspects of the listing, generating a low score of feedback for them and calculating this score in the overall rating, will we reduce the bias? I believe that the answer is yes, however in order to have an answer for these questions the research is planned as described in the next section.


\end{document}
