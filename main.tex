\documentclass[a4paper, 11pt]{article}
\usepackage{comment} % enables the use of multi-line comments (\ifx \fi) 
\usepackage{lipsum} %This package just generates Lorem Ipsum filler text. 
\usepackage{fullpage} % changes the margin

\begin{document}

\section*{Introduction}

Here will go the introduction. \textit{Preliminary example}: Sharing economy is an important phenomenon grown exponentially over the last decad. By utilizing information technology, individuals can reuse, distribute and share their excess capacity of goods and services. However, building trust in these online marketplaces is the key to the adoption of the sharing economy \cite{owen2014trust}. All the markets require some minimum amount of trust, but for Internet markets this is a particular challenge considering the fact that trades are typically anonymous, geographically dispread and executed sequentially. To incentivize trustworthiness, online markets employ reputation based "feedback systems" allowing traders to post information about past experiences. Examples of these systems include the traditional giants eBay and Amazon, to continue with the new big players such as Airbnb and Uber.

The proliferation of online feedback mechanisms is already changing people’s behavior in subtle
but important ways. Anecdotal evidence suggests that people now increasingly rely on opinions
posted on such systems in order to make a variety of decisions ranging from what movie to
watch to what stocks to invest on (Guernsey, 2000). Only five years ago the same people would
primarily base those decisions on advertisements or professional advice. (dellarocas) %(cite{dellarocas2003digitalization)}

(missing part)
Sentiment mining is the computational study of opinions, sentiments, evaluations, attitudes, appraisal, affects, views, emotions, subjectivity, etc., expressed in text. This text can be found in reviews, blogs, discussions, news, comments, feedback and so on. This analysis can also be mentioned as opinion mining and it refers to opinions gathered first from organizations internal data (customer feedback from emails, call centers), secondly from news and reports and thirdly based on the word-of-mouth on the Web (personal experiences, issues, reviews, social networking, blogs, forums etc). 

----> Sentiment analysis application? Business \& orgs, individuals ...

Since early 2000, sentiment analysis has grown to be one of the most
active research areas in natural language processing. It is also widely studied
in data mining, Web mining, and text mining. In fact, it has spread from
computer science to management sciences and social sciences due to its
importance to business and society as a whole. In recent years, industrial
activities surrounding sentiment analysis have also thrived. Numerous
startups have emerged. Many large corporations have built their own inhouse
capabilities. Sentiment analysis systems have found their applications
in almost every business and social domain.

With the explosive growth of social media (e.g., reviews, forum discussions,
blogs, micro-blogs, Twitter, comments, and postings in social network sites)
on the Web, individuals and organizations are increasingly using the content
in these media for decision making. Nowadays, if one wants to buy a
consumer product, one is no longer limited to asking one’s friends and
family for opinions because there are many user reviews and discussions in
public forums on the Web about the product. 
Each site
typically contains a huge volume of opinion text that is not always easily
deciphered in long blogs and forum postings. The average human reader will
have difficulty identifying relevant sites and extracting and summarizing the
opinions in them. Automated sentiment analysis systems are thus needed.\cite{liu2012sentiment}

information obtained from IDS
could be treated as a proxy measure of sociological or economical phenomena (e.g. Google
Trends). Moreover, there is a lack of statistical literature directly connected with estimation
problems related to new data sources. The results of the study suggest that the existing denitions and methodology, which are
valid for existing statistical data sources, should be adopted or revised to deal with new data
sources. \cite{berkesewicz2015representativeness}

While ‘manufactured’ trust-building arguments by Internet stores may be biased, text comments written by buyers in online auction marketplaces are likely to be objective, impartial, and unbiased. The tacit nature of feedback comments can convey the notion that a seller has previously acted in an outstanding fashion to pursue a buyer’s best interests, or acted in an abysmal manner to exploit a buyer’s vulnerabilities.
Most important, a distinct survey item asked the respondents to indicate how many feedback comments they examined for the seller they purchased from. 81\% reported examining 25 comments (one webpage), 5\% viewed 50 comments, 11\% more than 50 ones, and only 3\% did not examine any text comments. This suggests that the evaluation of the first 25 comments in a seller’s feedback profile is likely to provide representative information about each seller that is typically examined by buyers. \cite{pavlou2006institutional,pavlou2006nature} 

For online marketplaces to succeed and prevent a market of ‘lemons’, their feedback technologies must be able to
differentiate among sellers and generate price premiums for trustworthy ones (as returns to their superior feedback).
However, the literature has solely focused on positive and negative feedback ratings, alas ignoring the role of
feedback text comments. These text comments are proposed to convey useful tacit knowledge about a seller’s prior
transactions that cannot be described by simple positive and negative ratings. This study examines the ‘hidden’
content of feedback text comments and its role in building buyer’s trust in a seller’s benevolence and credibility. In
turn, benevolence and credibility are proposed to influence the price premiums that a seller receives from buyers. \cite{palou institutional}
This paper utilizes content analysis to analyze over 10,000 feedback text comments of 420 sellers in eBay’s online
auction marketplace, and match them with survey responses from 420 buyers that recently purchased products
from these sellers. These dyadic data show that feedback comments create price premiums for trustworthy sellers
by influencing buyer trust beliefs in a seller’s benevolence and credibility (even after controlling for the impact of
positive and negative ratings).

------


While the potential role of institutional feedback technologies has been demonstrated in the literature
(see Dellarocas 2003 for a review), the literature is potentially incomplete, reflected through the low variance
explained in price premiums (R2 = 20-30\%). This is because the literature has solely focused on quantitative
(positive and negative) feedback ratings,3 ignoring the “hidden” role of feedback text comments. We argue
that these text comments contain tacit knowledge that cannot be conveyed by positive and negative ratings.
More particularly, while the majority of feedback text comments is expected to be ordinary (confirming
the essence of positive and negative ratings), we argue that some feedback comments may contain evidence
of a seller’s extraordinary activities during its past transactions that are likely to stand out. Such extraordinary
comments may convey evidence of outstanding or abysmal benevolence or credibility in its past transactions,
and they are proposed to influence a buyer’s trust in the seller’s benevolence or credibility, respectively
(beyond the impact of feedback ratings). In addition to ordinary comments that simply denote that a transaction has been fulfilled properly, the
tacit nature of feedback comments can convey the notion that a seller has previously acted in an outstanding
fashion to pursue a buyer’s best interests, or acted in an abysmal manner to exploit a buyer’s vulnerabilities. First, this study identifies, conceptualizes, and operationalizes the role of feedback text comments as a
new trust-building means in online marketplaces. To the best of our knowledge, this is the first study to examine
the nature and hidden referral value of feedback text comments. Virtually all (97\%) of the study’s respondents
indicated having examined the feedback text comments of the sellers before transacting with them. An
important separation proposed and validated in this study is the distinct nature and role of (tacit) feedback text
comments and (explicit) feedback ratings, following the knowledge management literature. Whereas the
literature has focused on explicit feedback ratings, this study contributes to our better understanding of the role of institutional feedback technologies to build trust and predict price premiums. Most important, from a
predictive standpoint, feedback text comments help explain a substantial amount of variance in benevolence
and credibility (directly) and price premiums (indirectly), beyond existing variables (e.g., feedback ratings).
(IMPORTANT) Finally, despite the trust-building potential of feedback text comments, it is apparently difficult for buyers
to assess the meaning of numerous text comments (compared to ratings that can be concisely summarized). (pavlou)

-----

(yaakub 2012) Usually, a customer relationship management (CRM) system
can be used for dealing with structured data in a database. It can establish contacts and managing communications with
customers, analyze information about customers and make
campaigning to attract new customers. However, it is very hard
to integrate customers comments and feedback into the CRM
system because most of them are described in text opinions. In this paper, we present a new approach to integrate costumers opinion into a traditional CRM system. (CONCLUSIONS) (Add to the customer focus part) This research proposal has show the important of opinion mining in the CRM. The conventional CRM only emphasis information
from customer details such as personal information,
and characters’ of the customer to predict the future pattern
of purchasing the products. Besides that, the conventional
CRM only captured the product details from manufactures or
suppliers without information from the customers that using
the particular product to get the feedback from customers.
Our new architecture of CRM is combining the customers’
personal record, products record, and also the feedback from
the customer regarding the particular product that they already
used it. This new architecture is very significant to companies
and manufacturers to gather the best opinion from customers

\section{Customer Focus Theory}
\label{sec:CFTH}

When offering a service or product, it is essential to understand what consumers' preferences are and how can we shape the offerings accordingly to create value. Customer Focus Theory is one of Information Studies theories, which specifically offers a guide on how to put the focus on the customer. This theory should be explained according to the context of my research (where it is placed and adds value on the framework) as shown in Figure. 
--- Narrow down to "Receiving and utilization of customer feedback" and the accordance with the research topic.

% This is a requirement of Business Information Studies track: Base and support the thesis research on one (or more) of Information Studies theories. Link: http://is.theorizeit.org/wiki/Main_Page 

\section{Online Feedback systems}
\subsection{Principles of OFS and game theory}

Various mechanisms have been designed and utilized
in C2C auction markets to promote trust and reduce risk.
For instance,
eBay's “Feedback Forum” is a form of community enforcement.
In eBay, after each trade, both buyers and
sellers are encouraged to leave comments about their
trading partners based on their experience. Comments
about traders are kept under each trader's profile, and can
be accessed by everyone who visits eBay. This way, the
system tries to deter dishonest behavior by conveying
facts and opinions about past trades. Kollock [15]
conceptually summarizes online reputation systems and
concludes that their effectiveness to manage the risks of
unsecured trades seems to be impressive. Resnick et al.
[20] review the online reputation systems and argue that
the reputation systems appear to perform reasonably well
despite their theoretical and practical difficulties. \cite{yang2007effects}

(dellarocas2003digitalization) 

One of the most important new capabilities of the Internet relative to previous mass
communication technologies is its bi-directionality. Through the Internet, not only can
organizations reach audiences of unprecedented scale at a low cost, but also, for the first time in
human history, individuals can make their personal thoughts, reactions, and opinions easily
accessible to the global community of Internet users.
Word-of- mouth, one of the most ancient mechanisms in the history of human society, is being
given new significance by this unique property of the Internet. Online feedback mechanisms, also
known as reputation systems (Resnick et al., 2000), are using the Internet's bi-directional
communication capabilities in order to artificially engineer large-scale word-of-mouth networks
in which individuals share opinions and experiences on a wide range of topics, including
companies, products, services, and even world events. \textbf{The following section} lists several noteworthy examples
of such mechanisms in use today.
Perhaps the best-known application of online feedback mechanisms to date has been their use as
a technology for building trust in electronic markets. This has been motivated by the fact that
many traditional trust-building mechanisms, such as state-enforced contractual guarantees, tend
to be less effective in large-scale online environments (Kollock 1999). Online feedback
mechanisms have emerged as a viable mechanism for fostering cooperation among strangers in
such settings by ensuring that the behavior of a trader towards any other trader becomes publicly
known and may therefore affect the behavior of the entire community towards that trader in the
future.

Managers will therefore find that proper decision-making related to the implementation and use
of feedback mechanisms requires careful consideration, not only of their own actions, but also of
the likely responses of other players interconnected through them. Accordingly, the tools of
game theory play a prominent role in the study of these mechanisms. Given the importance of word-of-mouth networks in human society, reputation formation has
been extensively studied by economists using the tools of game theory. The game theory explained for OFS:
This makes the dynamics of reputation formation in
environments with imperfect monitoring quite complex indeed: an initial stage of reputation
formation (with potentially suboptimal payoffs) is followed by a stage where the long-run player
is able to occasionally “fool” short-run players and realize payoffs above his Stackelberg payoff,
followed by a stage where short-run players eventually learn the truth and the game reverts to its
static Nash equilibrium.
These dynamics have important repercussions for systems like eBay. According to the Cripps,
Mailath and Samuelson result, if eBay makes the entire feedback history of a seller available to
buyers and if an eBay seller stays on the system long enough, once he establishes an initial
reputation for honesty he will be tempted to cheat buyers every now and then. In the long term,
this behavior will lead to an eventual collapse of his reputation and therefore of cooperative behavior. The conclusion is that, if buyers pay attention to a seller’s entire feedback history,
eBay’s current mechanism fails to sustain long-term cooperation.

Most game theoretic models of reputation formation assume that stage game outcomes (or
imperfect signals thereof) are publicly observed. Online feedback mechanisms, in contrast, rely
on private monitoring of stage game outcomes and voluntary feedback submission. This
introduces two important new considerations: (a) ensuring that sufficient feedback is, indeed,
provided, and (b) inducing truthful reporting.

\subsection{Characteristics of most popular OFSs}
\label{subsec:popularOFS}

This subsection will explain how online feedback systems work. In addition, it will describe, categorize and point out the differences between the typical types of OFS used by the most successful online marketers (Ebay, Amazon, Yelp, TripAdvisor and Airbnb). 

\end{document}